Cząstka (jon) o masie \textbackslash{}( m \textbackslash{}) i ładunku
\textbackslash{}( q \textbackslash{}) wyemitowana ze źródła Z zostaje
przyspieszona napięciem \textbackslash{}( U \textbackslash{}) po czym
wlatuje w obszar jednorodnego pola magnetycznego \textbackslash{}( B
\textbackslash{}) prostopadłego do toru cząstki. (Pamiętaj, że symbol
\textbackslash{}( \textbackslash{}odot \textbackslash{}) oznacza wektor
skierowany przed płaszczyznę rysunku, a symbolem ⊗ oznaczamy wektor
skierowany za płaszczyznę rysunku). Pole magnetyczne zakrzywia tor
cząstki, tak że porusza się ona po półokręgu o promieniu
\textbackslash{}( R \textbackslash{}), po czym zostaje zarejestrowana w
detektorze (np. na kliszy fotograficznej) w odległości \textbackslash{}(
2R \textbackslash{}) od miejsca wejścia w pole magnetyczne.

Promień okręgu po jakim porusza się naładowana cząstka w polu
\textbackslash{}( B \textbackslash{}) obliczyliśmy w ostatnim ćwiczeniu

\textless{}div class="openaghmathjax-block" style="text-align:
center;"\textgreater{}\textless{}div
class="openaghmathjax-wzor"\textgreater{}\textbackslash{}(
\textbackslash{}begin\{equation\}\{R=\textbackslash{}frac\{\textbackslash{}mathit\{mv\}\}\{\{qB\}\}\}\textbackslash{}end\{equation\}
\textbackslash{})\textless{}/div\textgreater{}\textless{}/div\textgreater{}

gdzie \textbackslash{}( v \textbackslash{}) jest prędkością z jaką
porusza się cząstka. Tę prędkość uzyskuje ona dzięki przyłożonemu
napięciu \textbackslash{}( U \textbackslash{}). Zmiana energii
potencjalnej ładunku przy pokonywaniu różnicy potencjału
\textbackslash{}( U \textbackslash{}) jest równa energii kinetycznej
jaką uzyskuje ładunek

\textless{}div class="openaghmathjax-block" style="text-align:
center;"\textgreater{}\textless{}div
class="openaghmathjax-wzor"\textgreater{}\textbackslash{}(
\textbackslash{}begin\{equation\}\{\textbackslash{}mathit\{\textbackslash{}Delta
E\}\_\{\{k\}\}=\textbackslash{}mathit\{\textbackslash{}Delta
E\}\_\{\{p\}\}\}\textbackslash{}end\{equation\}
\textbackslash{})\textless{}/div\textgreater{}\textless{}/div\textgreater{}

lub

\textless{}div class="openaghmathjax-block" style="text-align:
center;"\textgreater{}\textless{}div
class="openaghmathjax-wzor"\textgreater{}\textbackslash{}(
\textbackslash{}begin\{equation\}\{\textbackslash{}frac\{\textbackslash{}mathit\{mv\}\^{}\{\{2\}\}\}\{2\}=\{qU\}\}\textbackslash{}end\{equation\}
\textbackslash{})\textless{}/div\textgreater{}\textless{}/div\textgreater{}

Stąd otrzymujemy wyrażenie na prędkość \textbackslash{}( v
\textbackslash{})

\textless{}div class="openaghmathjax-block" style="text-align:
center;"\textgreater{}\textless{}div
class="openaghmathjax-wzor"\textgreater{}\textbackslash{}(
\textbackslash{}begin\{equation\}\{v=\textbackslash{}sqrt\{\textbackslash{}frac\{2\{qU\}\}\{m\}\}\}\textbackslash{}end\{equation\}
\textbackslash{})\textless{}/div\textgreater{}\textless{}/div\textgreater{}

i podstawiamy je do równania ((Ruch naładowanych cząstek w polu
magnetycznym\textbar{}\#r22.3\textbar{}( 1 )))

\textless{}div class="openaghmathjax-block" style="text-align:
center;"\textgreater{}\textless{}div
class="openaghmathjax-wzor"\textgreater{}\textbackslash{}(
\textbackslash{}begin\{equation\}\{R=\textbackslash{}frac\{1\}\{B\}\textbackslash{}sqrt\{\textbackslash{}frac\{2\textbackslash{}text\{mU\}\}\{q\}\}\}\textbackslash{}end\{equation\}
\textbackslash{})\textless{}/div\textgreater{}\textless{}/div\textgreater{}

Ostatecznie po przekształceniu otrzymujemy

\textless{}div class="openaghmathjax-block" style="text-align:
center;"\textgreater{}\textless{}div
class="openaghmathjax-wzor"\textgreater{}\textbackslash{}(
\textbackslash{}begin\{equation\}\{m=\textbackslash{}frac\{R\^{}\{\{2\}\}B\^{}\{\{2\}\}q\}\{2U\}\}\textbackslash{}end\{equation\}
\textbackslash{})\textless{}/div\textgreater{}\textless{}/div\textgreater{}

Widzimy, że pomiar odległości (\textbackslash{}( 2R \textbackslash{})),
w jakiej została zarejestrowana cząstka pozwala na wyznaczenie jej masy
m.

Zakrzywianie toru cząstek w polu magnetycznym jest również
wykorzystywane w urządzeniach zwanych akceleratorami. Te urządzenia
służące do przyspieszania cząstek naładowanych, znalazły szerokie
zastosowanie w nauce, technice i medycynie. Przykładem akceleratora
cyklicznego jest cyklotron. O jego działaniu możesz przeczytać w module
((Cyklotron)).
